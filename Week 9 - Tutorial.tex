\documentclass[11pt, answers]{exam}
\usepackage{amsmath}
\usepackage{amsfonts}
\usepackage{amssymb}
\usepackage{diagbox}
\usepackage{color}
\usepackage{makecell}
\usepackage{multirow}
\usepackage{verbatim}

\usepackage{pdfpages}
\usepackage{setspace}
\usepackage{xspace}
\usepackage{tikz}\usepackage{pgflibraryplotmarks}
\usepackage[margin=1in]{geometry}
\usepackage{actuarialsymbol}
\usepackage{booktabs}
\usepackage{actuarialangle}
%%added by Liyuan
\newcommand{\tbl}{\textcolor{blue}}
\newcommand{\trd}{\textcolor{red}}
\newcommand{\bl}{\color{blue}}
\newcommand{\rd}{\color{red}}
\definecolor{darkblue}{rgb}{0.1,0.1,0.6}
\definecolor{darkgreen}{rgb}{0.1,0.5,0.1}
\newcommand{\tdg}{\textcolor{darkgreen}}
\newcommand{\tdb}{\textcolor{darkblue}}
\newcommand{\tpp}{\textcolor{purple}}
\newcommand{\tsb}{\textcolor{SkyBlue}}
\usepackage{tikz}
\usepackage{afterpage}
\usepackage[paper=A4,pagesize]{typearea}
\usepackage{lipsum}



\usetikzlibrary{arrows.meta}
\usetikzlibrary{decorations.pathreplacing}
% For an exam, single spacing is most appropriate
\singlespacing
% \onehalfspacing
% \doublespacing
\textheight 23.20cm
% For an exam, we generally want to turn off paragraph indentation
\parindent 0ex


\begin{document}
	
	% These commands set up the running header on the top of the exam pages
	\pagestyle{head}
	\firstpageheader{}{}{}
	\runningheader{ETC2430}{- Tutorials Week 8 -}{\ Page \thepage\ of \numpages}
	\runningheadrule
	
	\begin{center}
		\begin{tabular}{c}
			\large{\textbf{ETC2430}}  \\
			\large{\textbf{Actuarial Cash Flow Modeling}} \\
			\\
			\textbf{Tutorials}\\
			\\
			\textbf{Week 8} \\
			\textbf{Project appraisal}
		\end{tabular}\\
	\end{center}
	\rule[1ex]{\textwidth}{.1pt}
	
\bigskip\bigskip

\underline{\textbf{Question 1:}}\bigskip

In a particular country, which uses dollars as its national currency, price inflation has been running at 20\% pa for the last 20 years. Calculate the average annual real rate of return for each of the following investments:
\begin{itemize}
\item[(i)] A set of gold coins purchased for 14,000 on 1 January 2015 and sold for 20,000 on 31 December 2017.
\item[(ii)] A painting purchased for 3,000 on 1 March 2017 and sold for 3,200 on 1 September 2017.
\item[(iii)] A diamond purchased for 13,000 on 1 July 2016 and sold for 10,000 on 1 July 2018.
\item[(iv)] A statuette purchased for 7,500 on 1 November 2010 and sold for 19,000 on 31 December 2017.
\end{itemize}
\bigskip\bigskip


\underline{\textbf{Solution:}}\bigskip


\bigskip\bigskip
\underline{\textbf{Question 2:}}\bigskip

On 1 January 2015 an investor purchased 10,000 nominal of a stock that pays coupons half-yearly on 30 June and 31 December each year at the rate of 6\% pa and is redeemable at par on 31 December 2027. The investor is liable for income tax at the rate of 40\%, but is not liable for capital gains tax. Calculate the price paid by the investor in order to achieve a net redemption yield of 5\% pa effective.
\bigskip\bigskip


\underline{\textbf{Solution:}}\bigskip


\bigskip\bigskip
\underline{\textbf{Question 3:}}\bigskip

A fixed-interest security with a coupon rate of 6\% pa payable half-yearly in arrears is purchased by an investor who is subject to capital gains tax at a rate of 30\%. The fixed-interest security is redeemable at par after 15 years. The price paid is such that the investment provides a gross redemption yield of 10\% pa effective. Calculate the amount of capital gains tax payable by the investor per 100 nominal.

\bigskip\bigskip
\underline{\textbf{Solution:}}\bigskip


\bigskip\bigskip

\underline{\textbf{Question 4:}}\bigskip

A stock with a term of 9.5 years has a coupon rate of 5\% pa payable half-yearly in arrears and is redeemable at 105\%. An investor who is not subject to tax purchases 100 nominal of the stock for 85. Calculate the yield obtained by the investor.

\bigskip\bigskip

\underline{\textbf{Solution:}}\bigskip

\underline{\textbf{Question 5:}}\bigskip

An investor purchases a bond 6 months after issue. The bond will be redeemed at 105\% eight
years after issue and pays coupons of 4\% per annum annually in arrears. The investor pays tax of 25\% on income and 15\% on capital gains.
\begin{itemize}
\item[(i)] Calculate the purchase price of the bond per £100 nominal to provide the investor with a rate of return of 5\% per annum effective. 
\item[(ii)] The real rate of return expected by the investor from the bond is 2\% per annum effective. Calculate the annual rate of inflation expected by the investor.
\end{itemize}
\bigskip\bigskip

\underline{\textbf{Solution:}}\bigskip


\underline{\textbf{Question 6:}}\bigskip

An investor purchased a government bond on 1 January in a particular year. The bond pays
coupons of 6\% pa six monthly in arrears on 30 June and 31 December. The bond is due to be redeemed at 105\% 11 years after the purchase date.
The investor pays income tax at the rate of 23\% on 1 April on any coupon payments received in the previous year (1 April to 31 March), and also pays capital gains tax on that date at the rate of 40\% on any capital gains realised in the previous year.
\begin{itemize}
\item[(i)] Calculate the price paid for 100 nominal of the bond, given that the investor achieves a net yield of 5\% pa effective interest. 
\item[(ii)] Without doing any further calculations, explain how and why your answer to (i) would alter if tax were collected on 1 October instead of 1 April each year.
\end{itemize}
\bigskip\bigskip

\underline{\textbf{Solution:}}\bigskip


\underline{\textbf{Question 7:}}\bigskip

An investor purchases 100 nominal of a fixed-interest stock, which pays coupons of 7\% pa
half-yearly in arrears. The stock is redeemable at par and can be redeemed at the option of the borrower at any time between 5 and 10 years from the date of issue. The investor is subject to tax at the rate of 40\% on income and 25\% on capital gains.
\begin{itemize}
\item[(i)] Calculate the maximum price that the investor should pay in order to obtain a net yield of at least 6\% pa. 
\item[(ii)] Given that this was the price paid by the investor, calculate his net annual running yield, convertible half-yearly.
\end{itemize}
\bigskip\bigskip

\underline{\textbf{Solution:}}\bigskip


\underline{\textbf{Question 8:}}\bigskip

An equity pays half-yearly dividends. A dividend of d per share is due in exactly 3 months’ time. Subsequent dividends are expected to grow at a compound rate of g per half-year forever.
\begin{itemize}
\item[(i)] If i denotes the annual effective rate of return on the equity, show that P , the price per share, is given by:
$$P=\frac{d(1+i)^{1/4}}{(1+i)^{1/2}-(1+g)}$$
\item[(ii)] The current price of the share is 3.60, dividend growth is expected to be 2\% per half-year and the next dividend payment in 3 months is expected to be 12p.

Calculate the expected annual effective rate of return for an investor who purchases the share.
\end{itemize}
\bigskip\bigskip

\underline{\textbf{Solution:}}\bigskip

\underline{\textbf{Question 9:}}\bigskip

An ordinary share pays dividends on each 31 December. A dividend of 35p per share was paid on
31 December 2017. The dividend growth is expected to be 3\% in 2018, and a further 5\% in 2019. Thereafter, dividends are expected to grow at 6\% per annum compound in perpetuity.
\begin{itemize}
\item[(i)] Calculate the present value of the dividend stream described above at a rate of interest of 8\% per annum effective for an investor holding 100 shares on 1 January 2018. 
\end{itemize}
\bigskip

An investor buys 100 shares for 17.20 each on 1 January 2018. He expects to sell the shares for 18 on 1 January 2021.
\begin{itemize}
\item[(ii)] Calculate the investor’s expected real rate of return.
\end{itemize}
You should assume that dividends grow as expected and use the following values of the inflation index:
\begin{table}[h]
\centering
\begin{tabular}{ccccc}

Year: & 2018 & 2019 & 2020 & 2021 \\
Inflation index at start of year: & 110.0 & 112.3 & 113.2 & 113.8
\end{tabular}
\end{table}
\bigskip

\underline{\textbf{Solution:}}\bigskip


\underline{\textbf{Question 10:}}\bigskip

 An index-linked zero-coupon bond was issued on 1 January 2013 for redemption at par on
31 December 2017. The redemption payment was linked to a price inflation index with a 6-month time lag. The value of the price index on different dates is given below:
\begin{table}[h]
\centering
\begin{tabular}{cccc}
Date & Index & Date & Index  \\
01.01.12 & 144 &01.01.16 & 181 \\
01.07.12 & 148 &01.07.16 & 182 \\
01.01.13 & 155 &01.01.17 & 188 \\
01.07.13 & 160 &01.07.17 & 193 \\
01.01.14 & 162 &01.01.18 & 201 \\
01.07.14 & 168 &~&~\\	
01.01.15 & 175 &~&~\\
01.07.15 & 177 &~&~\\
\end{tabular}
\end{table}
Calculate the annual effective money and real rates of return obtained by an investor who purchased 10,000 nominal of the stock on 1 January 2015 for 10,250 and held it until redemption.


\underline{\textbf{Solution:}}\bigskip

\underline{\textbf{Question 11:}}\bigskip

On 25 October 2013 a certain government issued a 5-year index-linked stock. The stock had a
nominal coupon rate of 3\% per annum payable half-yearly in arrears and a nominal redemption price of 100\%. The actual coupon and redemption payments were index-linked by reference to a retail price index as at the month of payment.

An investor, who was not subject to tax, bought 10,000 nominal of the stock on 26 October 2017. The investor held the stock until redemption.

You are given the following values of the retail price index:
\begin{table}[h]
\centering
\begin{tabular}{|c|c|c|c|c|}
	\hline
& 2013 & - & 2017 &2018\\
\hline
April & -&-&-&171.1\\
October & 149.2&-&169.4&173.8\\
\hline
\end{tabular}
\end{table}
\begin{itemize}
\item[(i)] Calculate the coupon payment that the investor received on 25 April 2018 and the coupon and redemption payments that the investor received on 25 October 2018. [3]
\item[(ii)] Calculate the purchase price that the investor paid on 25 October 2017 if the investor achieved an effective real yield of 3.5\% per annum effective on the investment.
\end{itemize}
\bigskip

\underline{\textbf{Solution:}}\bigskip


\underline{\textbf{Question 12:}}\bigskip

On 15 May 2017 the government of a country issued an index-linked bond of term 10 years.
Coupons are payable half-yearly in arrears, and the annual nominal coupon rate is 8\%. The nominal redemption price is 102\%.
\medskip

Coupon and redemption payments are indexed by reference to the value of a retail price index with a time lag of 6 months. The retail price index value in November 2016 was 185 and in May 2017 was 190.

\medskip
The issue price of the bond was such that, if the retail price index were to increase continuously at a rate of 2\% pa from May 2017, a tax-exempt purchaser of the bond at the issue date would obtain a real yield of 3\% pa convertible half-yearly.

\medskip
Show that the issue price of the bond is 146.85 per 100 nominal.

\bigskip

\underline{\textbf{Solution:}}\bigskip	


\newpage

{\Large \textbf{Solutions:}}
\begin{itemize}
	\item Question 1: (i) 3,996.16 (ii)3.71 (iii)18.9\%
	\item Question 2: 135
	\item Question 3: 6.84\%
	\item Question 4: (i)15.5 (ii)14.662 million
	\item Question 5: (i)45,484 (ii) DPP does not fall within the first 10 years (iii)16.618 years
\end{itemize}

\end{document}
